\documentclass[12pt]{article}
%%%%%%%%%%%%%%%%%%%%%%%%%%%%%%%%%%%%%%%%%%%%%%%%%%%%%%%%%%%%%%%%%%%%%%%%%%%%%%%%%%%%%%%%%%%%
\def\TheAuthor{Michael Monkenbusch}                           % Author name
\def\TheTitle{Labbook Notes}                                  % Title of contribution
\def\TheHeading{Labbook Notes}                                % Short running title
\def\TheInstitute{JCNS (former IFF)}                          % Institute
\def\TheAddress{FZJ}                                          % Place e.g. University ...
\def\ThePart{}                                                % Part of book
\def\TheChapter{I}                                            % Chapter of contribution
%%%%%%%%%%%%%%%%%%%%%%%%%%%%%%%%%%%%%%%%%%%%%%%%%%%%%%%%%%%%%%%%%%%%%%%%%%%%%%%%%%%%%%%%%%%%

\usepackage{mylabbook}                                        % Style file for layout, derived from Ferienschule_07.sty
\usepackage{float}                                            % Figure placement float
\usepackage{etex}
\usepackage{epsfig,color}
\usepackage{makeidx}
\usepackage{cases}

% General translations
\newcommand{\Aa}{\rm{\AA}}
%
%%%%%%%%%%%%%%%%%%%%%%%%%%%%%%%%%%%%%%%%%%%%%%%%%%%%%%%%%%%%%%%%%%%%%%%%%%%%%%%%%%
% Auxiliaries for theory description
%%%%%%%%%%%%%%%%%%%%%%%%%%%%%%%%%%%%%%%%%%%%%%%%%%%%%%%%%%%%%%%%%%%%%%%%%%%%%%%%%%
% Headersection
\newcommand{\thname}[2]{{\Large \bf #1} \vskip 15pt {#2} \vskip 20pt}

% Parameter Section
\newcommand{\parameters}{\vskip 15pt {\bf Parameters:} \vskip 5pt \begin{tabular}{lllll} \hline name & symbol & meaning & units & fit \\ \hline}
\newcommand{\eparameters}{\hline \end{tabular}}
\newcommand{\parnam}[5]{#1 & #2 & #3 & #4 & #5\\}
% Description section

\newcommand{\thdescript}[1]{{\vskip 15pt \bf General description} \vskip 10pt #1}

% Computation
\newcommand{\algorithm}[1]{\vskip 15pt {\bf Algorithm} \vskip 15pt \begin{eqnarray} #1 \end{eqnarray}}

% Expected Fileparameters
\newcommand{\expectedfileparameters}{\vskip 15pt {\bf Expected file parameters:} \vskip 5pt \begin{tabular}{lllll} \hline name & symbol & meaning & units & optional \\ \hline}
\newcommand{\eexpectedfileparameters}{\hline \end{tabular}}
\newcommand{\efileparnam}[5]{#1 & #2 & #3 & #4 & #5\\}

% created File Parameters
\newcommand{\fileparameters}{\vskip 15pt {\bf Associated file parameters:} \vskip 5pt \begin{tabular}{llll} \hline name & symbol & meaning & units \\ \hline}
\newcommand{\efileparameters}{\hline \end{tabular}}
\newcommand{\fileparnam}[4]{#1 & #2 & #3 & #4 \\}


%%%%%%%%%%%%%%%%%%%%%%%%%%%%%%%%%%%%%%%%%%%%%%%%%%%%%%%%%%%%%%%%%%%%%%%%%%%%%%%%%%%%%%%%%%%
%%%%%%%%%%%%%%%%%%%%%%%%%%%%%%%%%%%%%%%%%%%%%%%%%%%%%%%%%%%%%%%%%%%%%%%%%%%%%%%%%%%%%%%%%%%
%%%%%%%%%%%%%%%%%%%%%%%%%%%%%%%%%%%%%%%%%%%%%%%%%%%%%%%%%%%%%%%%%%%%%%%%%%%%%%%%%%%%%%%%%%%
\makeindex
\begin{document}
\MakeTitel           %%% Displays title, author name, etc.
\tableofcontents     %%% Displays table of contents
\newpage
\setlength{\parskip}{0.3cm}
%%%%%%%%%%%%%%%%%%%%%%%%%%%%%%%%%%%%%%%%%%%%%%%%%%%%%%%%%%%%%%%%%%%%%%%%%%%%%%%%%%%%%%%%%%%

\thname{th\_grotmod}{Gaussian u*u models}
%-------------------------------------
\parameters
\parnam{intensity}{a0}{prefactor}{y-units}{yes}
\eparameters
%
\expectedfileparameters
\efileparnam{q}{$Q$}{momentum transfer}{$\rm \AA^{-1}$}{no}
\efileparnam{ga1inten}{$a_1$}{gaussian resolution component amplitude} {y-units}{no}
\efileparnam{ga1width}{$w_1$}{gaussian resolution component width    } {x-units}{no}
\efileparnam{ga1width}{$\omega_1$}{gaussian resolution component center    } {x-units}{no}
\eexpectedfileparameters
%
\fileparameters
\fileparnam{p1}{p1}{parameter 1}{xy-units}
\efileparameters

\algorithm{ 
&F_{diro}(Q,t,r_1,r_2) = \exp(-D_{\rm cm} \, Q^2 \, t) \; { \sum_m f_m { \sum_l^{l_{\rm max}} (2l+1) A_l(Q,r_{1,m},r_{2,m}) \, e^{-l(l+1)D_r\,t}  } }  
\cr\cr
& A_n(Q,r_1,r_2) = \int_{r_1}^{r_2} {j_n(Q r)^2 \, r^2 \, dr} / \int_{r_1}^{r_2} {r^2 \, dr}  = \cr
&               \frac{3}{2(r_2^3-r_1^3)}\left (
                            r_1^3 \, \left [j_{n-1}(Q r_1) \, j_{n+1}(Q r_1) \,-\,j_n(Q r_1)^2  \right ]
                            -r_2^3 \, \left [j_{n-1}(Q r_2) \, j_{n+1}(Q r_2) \,-\,j_n(Q r_2)^2  \right ]
                            \right ) 
\cr
\cr
&F_{dsph}(Q,t,R) =  
                    \left ( \frac{3 j_1(QR)}{QR} \right )^2 +
                  6 \,{ \sum_{(l,\nu) \ne (0,0)}^{l_{\rm max}, \nu_{\rm max}} 
{ 
\left {
 \left [ \frac{ Q R\, j_{l+1}(QR) - l j_{l}(QR) }{  (QR)^2-z_{l,\nu}^2} \right ]^2 \, \frac{(2l+1) \, z_{l,\nu}^2}{  z_{l,\nu}^2 - l(l+1) }
\right }
 \, e^{-z_{l,\nu}^2\,D_s/R^2\,t}  }    

\cr\cr
&S_{\rm inc}(Q,t) = \exp(-D_{\rm cm} \, Q^2 \, t) \; { \sum_m f_m \,[(1-\phi_m)+\phi_m F_{dsph}(Q,t,R)] \, \cr&{ \sum_l^{l_{\rm max}} (2l+1) A_l(Q,r_{1,m},r_{2,m}) \, e^{-l(l+1)D_r\,t}  } }  

}
}

\section{Model}
%--------------
\label{sec:model}
%................
In order to describe the incoherent quasielastic scattering the following model scenario 
has been assumed.
\begin{itemize}
\item The whole molecule undergoes translational diffusion with $D_t$.
\item The molecule shows {\it isotropic} rotational diffusion $D_r$, the contributions
of various protons within the molecule depends on their radius of rotation (here distance from
the center-of-mass). 
\item A fraction of the protons has additional mobility that is in a coarse way modeled by
the assumption that they diffusion in a shperical confinement (using the well known model
of Dianoux and Volino \cite{volino80vo}. 
\end{itemize}
The convolution with the instrumental resolution is performed by multiplying the time function
of the model with the Fourier-tranform of a representation of the resolution function in the form
of a sum of Gaussians. The resuklting time function then has a limited carrier and numerical back-transform
yields the convoluted spectrum.
Determinig the Gaussian resolution represenation by fitting with a function that compares the integral
of the Gaussian component over a time-channel width (i.e. erf..) it is also ensured that no
width offset due to the finite channel width enters the final results (only relevant for marginal
broadening of the elastic line by a fraction of its width respectively an amount comarable to the
nominal energy transfer of one or a few channel widths).  
%
% --> th_dirosp
%
%        sumtf = 0
%        do i=1,3
%         if(rro(i).gt.rri(i)) then
%           do l=0,lmax
%             sumtf=sumtf + fr(i) *
%     *             (2*l+1)*radial_bsjn2_integral(l,qc,rri(i),rro(i))*
%     *             exp(-l*(l+1)*rotdiff*t)
%           enddo
%         endif
%        enddo
%       
%        sumtf = sumtf * exp(-diffcm*qc*qc*t)
%
is derived from the expression for rotational diffusion \cite{sears66seaa}. 
\begin{equation}
\label{eq:dirosp}
F_{diro}(Q,t,r_1,r_2) = \exp(-D_{\rm cm} \, Q^2 \, t) \; { \sum_m f_m { \sum_l^{l_{\rm max}} (2l+1) A_l(Q,r_{1,m},r_{2,m}) \, e^{-l(l+1)D_r\,t}  } }  
\end{equation}
with $f_m$ the fraction of protons in shell $m$ that extends from $r=r_{1,m} \cdots r_{2,m}$, $D_{\rm cm}$ the center of mass
diffusion of the whole globule and $D_r$ its (isotropically averaged) rotational diffusion constant.
% 
%      function radial_bsjn2_integral(n,q,r1,r2)
%!     -----------------------------------------
%!
%!     Integral over a shell from r1..r2 of bsjn(n,q*r)**2  as is needed in the rotational diffusion
%!     modelling for a rotating protein in incoherent approximation
%!     The integral is normalized to the shell volume
%!
%!     i.e. integral_[r1..r2](bsjn(x,q*r)**2 *r**2 * dr) / integral_[r1..r2](r**2 *dr) 
%!
%      implicit none
%      double precision              :: radial_bsjn2_integral
%      double precision, intent(in)  :: q, r1, r2
%      integer         , intent(in)  :: n
%    
%      double precision              :: bsjn
%
%      radial_bsjn2_integral = -(0.5d0)*R1**3*bsjn(n  , R1*q)**2                 &
%                              +(0.5d0)*R1**3*bsjn(n-1, R1*q)  *bsjn(n+1, R1*q)  &
%                              +(0.5d0)*R2**3*bsjn(n  , R2*q)**2                 &
%                              -(0.5d0)*R2**3*bsjn(n-1, R2*q)  *bsjn(n+1, R2*q) 
%
%
%!     normalizing to volume
%      
%      radial_bsjn2_integral =  radial_bsjn2_integral*3/(r2**3-r1**3)
%
%      return
%      end function  radial_bsjn2_integral
%
%
\begin{eqnarray}
\label{eq:bsjn2}
& A_n(Q,r_1,r_2) = \int_{r_1}^{r_2} {j_n(Q r)^2 \, r^2 \, dr} / \int_{r_1}^{r_2} {r^2 \, dr}  = \cr
&               \frac{3}{2(r_2^3-r_1^3)}\left (
                            r_1^3 \, \left [j_{n-1}(Q r_1) \, j_{n+1}(Q r_1) \,-\,j_n(Q r_1)^2  \right ]
                            -r_2^3 \, \left [j_{n-1}(Q r_2) \, j_{n+1}(Q r_2) \,-\,j_n(Q r_2)^2  \right ]
                            \right ) 
\end{eqnarray}
%
%
at least for a fraction of the protons additional degrees of freedom are assumed and modelled by 
the assumption of confined diffusion in a spheres of radius $R$ that are attached at some point of the globule.
The well known expression for diffusion inside a sphere \cite{volino80vo} expressed as intermediate scattering
function is
\begin{equation}
\label{eq:diavo}
F_{dsph}(Q,t,R) =  
                    \left ( \frac{3 j_1(QR)}{QR} \right )^2 +
                  6 \,{ \sum_{(l,\nu) \ne (0,0)}^{l_{\rm max}, \nu_{\rm max}} 
{ 
\left {
 \left [ \frac{ Q R\, j_{l+1}(QR) - l j_{l}(QR) }{  (QR)^2-z_{l,\nu}^2} \right ]^2 \, \frac{(2l+1) \, z_{l,\nu}^2}{  z_{l,\nu}^2 - l(l+1) }
\right }
 \, e^{-z_{l,\nu}^2\,D_s/R^2\,t}  }    
}
\end{equation}
with
$z_{l,\nu}$ the zeros of the equation $0=j_l(z)-z\,j_{l+1}(z)$, typical values used here are $l_{\rm max} = 30 \cdots 40$ and 
$z_{\rm max} = 30$.
The combined model assuming restricted additional proton motions in some shells of the globule reads:
\begin{equation}
\label{eq:dconv}
S_{\rm inc}(Q,t) = \exp(-D_{\rm cm} \, Q^2 \, t) \; { \sum_m f_m \,[(1-\phi_m)+\phi_m F_{dsph}(Q,t,R)] \, { \sum_l^{l_{\rm max}} (2l+1) A_l(Q,r_{1,m},r_{2,m}) \, e^{-l(l+1)D_r\,t}  } }  
\end{equation}
where $\phi_m$ denotes the fraction of mobile protons in shell $m$ and $\sum_m f_m = 1$.
%

{\color{red}{
Equations \ref{eq:diavo} and \ref{eq:dconv} describe the data and represent the physical situatuion reasonably well,
however, they still ignore the possible contribution of large scale domain motions.
This may be done by replacing the expression for diffusion in a shpere (Eq. \ref{eq:diavo}) by a 
simpler expression pertaining to a soft confinement such that the protons correlation is Gaussian.
This approach has been worked out by Volino et al. \cite{volino06vo} and compared to the behaviour
of Eq. \ref{eq:diavo}. Even if the details differ a bit the salient features of both models match
if the radius $R^2\simeq 5 \langle u^2 \rangle $ and the diffusion constant inside the sphere
relates to the relaxation time of particle correlation $\tau = \langle u^2 \rangle / D_s$ 
towards the Gaussian with width $\sqrt{\langle u^2 \rangle}$.
The expression for the corresponding intermediate scattering function is simply
\begin{equation}
\label{eq:gsp}
F_{\rm gsp}(Q,t) = \exp\left \{ -Q^2 \frac{R^2}{5}[1-e^{-t\, D_s/\langle u^2 \rangle}] \right \}
\end{equation}     
%
Now this formulation allows the additional incorporation of the effects of a large scale domain
motion. Assuming that the domain motion alone again corresponds to Gaussian distribution with
a characteristic correlation time $\tau_D$ and a distribution withd $\langle a^2 \rangle$ that correspond to
the amplitude of the large scale dynamics. The associated motions of the affected protons 
is considered as one dimensional and its amplitude depends on the location of the proton in
the domains. Combinig the motions is performed by adding the associated mean squared displacements,
i.e. in one direction $\langle u^2 \rangle (1-\exp [-t D_s /\langle u^2 \rangle ])$ is replaced by
\begin{equation}
\label{eq:ado}
A = \langle u^2 \rangle (1-\exp[-t D_s /\langle u^2 \rangle]) + \langle a^2 \rangle (1-\exp[-t/\tau_D])
\end{equation}
With $B= \langle u^2 \rangle (1-\exp[-t D_s /\langle u^2 \rangle])$ the intermediate scattering
function of the combined motion after angular averaging the reads:
\begin{equation}
\label{eq:adoplus}
F_{\rm gsp+} (Q,t) = \frac{\sqrt{\pi}}{2} \; \frac{{\rm erf}\left(\sqrt{Q^2(A-B)}\right)\,e^{-Q^2\,B}}{\sqrt{Q^2(A-B)}}
\end{equation}
If the domain motion is rather considered to be two-dimensional the roles of $A$ and $B$ just have to be
exchanged.
Within this model the overall scattering function is obtained as
\begin{equation}
\label{eq:dconv2}
S_{\rm inc}(Q,t) = \exp(-D_{\rm cm} \, Q^2 \, t) \; {\sum_{m,n} f_{m,n} \,F_{\rm gsp+}(Q,t,\langle u^2 \rangle_{m,n}, \langle a^2 \rangle_{m,n}) 
\, { \sum_l^{l_{\rm max}} (2l+1) A_l(Q,r_{1,m},r_{2,m}) \, e^{-l(l+1)D_r\,t}  } }  
\end{equation}
instead of Eq. \ref{eq:dconv}. Where $f_{m,n}$ is the fraction of protons of a class $n$ within shell $m$.   

At larger momentum transfer a further simplification can be made by observing that the rotation diffusion
part of the proton mean square displacement locally is equivalent to a 2D diffusion with a diffusion
constant of $D_{\rm 2D,rot} = f\,D_{\rm rot} \, r^2$. This means that instead of multiplying with the
Sears expansion (last sum of Eq. \ref{eq:dconv2}) simply the contribution
$B_{\rm rot} =  f\,D_{\rm rot} \, r^2 \, Q^2 t$ has to be added to $B$ in Eq. \ref{eq:adoplus}.

}}


Resolution convolution is performed by multiplication of the intermediate scattering function with the
Fourier transform of the resolution function. This Fourier transform is obtained by using a representation 
of the resolution function in terms of Gaussians
\begin{equation}
\label{eg:gares}
{\cal R}(Q,\omega) = \sum_{i=1}^{N} { a_i \, e^{-(\omega-\omega_i)^2/w_i^2} }
\end{equation}
with the Fourier-representation
\begin{equation}
\label{eg:garesF}
{\cal R}(Q,\omega) = \sum_{i=1}^{N} {\int \frac{a_i \, e^{-(w_i \, t)^2/4}}{ 2 \pi^{3/2} \, w_i }\, \cos([\omega-\omega_i]\,t) d\omega}
\end{equation}
typically the components of the Gaussian representations have a rather limited extension in
the time domain such that the numerical Fouriertransform of the product with the intermediate
scattering function can be restricted to a corresponding maximum time, thus yielding an 
efficient computation of the expected experimentally observed spectrum
\begin{equation}
\label{eq:conv1}
I(Q,\omega) = \sum_{i=1}^{N} { \int_0^{t_{\max} = W_t/w_i} {
\frac{a_i \, e^{-(w_i \, t)^2/4}}{ 2 \pi^{3/2} \, w_i } \, \cos([\omega-\omega_i]\, t) \, S(Q,t) \, dt 
}}
\end{equation}
where the coefficient $W_t$ used for the upper time limit is selected such that the integral
contributions beyond that point are sufficiently reduced, 
%here a value of 5 is used sich that the integrand is reduced with a factor of less than $2 \times 10^{-3}$ 
here a value of 9 is used sich that the integrand is reduced with a factor of less than $2 \times 10^{-9}$ 
due to the resolution
factor (if reqired the factor may easily be increased without causing too much extra computational effort). 
The procedure as outline above captures the salient features use for fitting the results,
however, to cope with effects pertaining to the finite width of one energy/frequency bin
$\Delta \omega$, which will play a role if broadening effects in the range of $\Delta \omega$
are to be considered, the final form of the method takes account for these effects by the
following modifications.
1. the resolution is fitted with the integral of the Gaussian over the channel width 
%
$ \sum_{i=1}^{N} { a_i \{ \sqrt{\pi/4}\, w_i/\Delta \omega \} \,(
{\rm erf}([2\omega+\Delta \omega]/2w_i)-{\rm erf}([2\omega-\Delta\omega]/2w_i)) $.
2. when computing the estimation of the measured intensity $I(Q,\omega)$ the same
integration over a channel width is performed, which effectively is achieved by
replacing Eq. \ref{eq:conv1} by
\begin{equation}
\label{eq:conv1}
I(Q,\omega) = \sum_{i=1}^{N} { \int_0^{t_{\max} = W_t/w_i} {
\frac{a_i \, e^{-(w_i \, t)^2/4}}{ 2 \pi^{3/2} \, w_i } \, 
\frac{ 
\sin(\{-[\omega-\omega_i]+\Delta\omega/2\}\, t)+\sin(\}[\omega-\omega_i]+\Delta\omega/2\}\, t)
}{
t \, \Delta\omega
} 
\, S(Q,t) \, dt 
}}
\end{equation}
 
%        strex_kernel_d0= 
%     *  sumtf * 
%     *  exp(-1d0/4d0*(str_delta*t)**2) * 
%     *  (sin(-t*Omega+0.5d0*t*xwidth)+sin(t*Omega+0.5d0*t*xwidth))/    !! this replaces cos(t*Omega)
%     *  (t*xwidth) * str_delta                                         !! in order to yield the integral
%                                                                       !! over one channel width in omega
This ensures proper treatment of spectral features with variations within the 
bin width of the histogramm of count results. For virtually any conceivable model for the
intermediate scattering function $S(Q,t)$ the numerical integration using an adaptive
method is reliable and fast. Due to the inclusion of the resolution convolution any
cutoff effects are avoided as well as the typical artefacts of replacing the integration 
by an FFT sum. 

Figure \ref{fig:resex} illustrates the representation of the IN5 resolution by two
Gaussians and a small constant background. The background is removed by using the
two Gaussians only fort convolution.
%fffffffffffffffffffffffffffffffffffffffffffffffffffffffffffffffffffffffffff
\begin{figure}[ht]
	\centering
  \includegraphics[width=0.5\textwidth, angle=90]{resolution_example.eps}
	\caption{IN5 resolution for a wavelength of $15\,{\rm \AA}$ at a scattering angle of $110^0$.
                 The data are the scattering from a 1~mm vanadium plate. The lines correspond to 
       This may be done by replacing the expression for diffusion in a shpere (Eq. \ref{eq:diavo}) by a 
simpler expression pertaining to a soft confinement such that the protons correlation is Gaussian.
This approach has been worked out by Volino et al. \cite{volino06vo} and compared to the behaviour
of Eq. \ref{eq:diavo}. Even if the details differ a bit the salient features of both models match
if the radius $R^2\simeq 5 \langle u^2 \rangle $ and the diffusion constant inside the sphere
relates to the relaxation time of particle correlation $\tau = \langle u^2 \rangle / D_s$ 
towards the Gaussian with width $\sqrt{\langle u^2 \rangle}$.
The expression for the corresponding intermediate scattering function is simply
\begin{equation}
\label{eq:gsp}
F_{\rm gsp}(Q,t) = \exp\left \{ -Q^2 \frac{R^2}{5}[1-e^{-t\, D_s/\langle u^2 \rangle}] \right \}
\end{equation}     
%
Now this formulation allows the additional incorporation of the effects of a large scale domain
motion. Assuming that the domain motion alone again corresponds to Gaussian distribution with
a characteristic correlation time $\tau_D$ and a distribution withd $\langle a^2 \rangle$ that correspond to
the amplitude of the large scale dynamics. The associated motions of the affected protons 
is considered as one dimensional and its amplitude depends on the location of the proton in
the domains. Combinig the motions is performed by adding the associated mean squared displacements,
i.e. in one direction $\langle u^2 \rangle (1-\exp [-t D_s /\langle u^2 \rangle ])$ is replaced by
\begin{equation}
\label{eq:ado}
A = \langle u^2 \rangle (1-\exp[-t D_s /\langle u^2 \rangle]) + \langle a^2 \rangle (1-\exp[-t/\tau_D])
\end{equation}
With $B= \langle u^2 \rangle (1-\exp[-t D_s /\langle u^2 \rangle])$ the intermediate scattering
function of the combined motion after angular averaging the reads:
\begin{equation}
\label{eq:adoplus}
F_{\rm gsp+} (Q,t) = \frac{\sqrt{\pi}}{2} \; \frac{{\rm erf}\left(\sqrt{Q^2(A-B)}\right)\,e^{-Q^2\,B}}{\sqrt{Q^2(A-B)}}
\end{equation}
If the domain motion is rather considered to be two-dimensional the roles of $A$ and $B$ just have to be
exchanged.
Within this model the overall scattering function is obtained as
\begin{equation}
\label{eq:dconv2}
S_{\rm inc}(Q,t) = \exp(-D_{\rm cm} \, Q^2 \, t) \; {\sum_{m,n} f_{m,n} \,F_{\rm gsp+}(Q,t,\langle u^2 \rangle_{m,n}, \langle a^2 \rangle_{m,n}) 
\, { \sum_l^{l_{\rm max}} (2l+1) A_l(Q,r_{1,m},r_{2,m}) \, e^{-l(l+1)D_r\,t}  } }  
\end{equation}
instead of Eq. \ref{eq:dconv}. Where $f_{m,n}$ is the fraction of protons of a class $n$ within shell $m$.   

At larger momentum transfer a further simplification can be made by observing that the rotation diffusion
part of the proton mean square displacement locally is equivalent to a 2D diffusion with a diffusion
constant of $D_{\rm 2D,rot} = f\,D_{\rm rot} \, r^2$. This means that instead of multiplying with the
Sears expansion (last sum of Eq. \ref{eq:dconv2}) simply the contribution
$B_{\rm rot} =  f\,D_{\rm rot} \, r^2 \, Q^2 t$ has to be added to $B$ in Eq. \ref{eq:adoplus}.

}}


Resolution convolution is performed by multiplication of the intermediate scattering function with the
Fourier transform of the resolution function. This Fourier transform is obtained by using a representation 
of the resolution function in terms of Gaussians
\begin{equation}
\label{eg:gares}
{\cal R}(Q,\omega) = \sum_{i=1}^{N} { a_i \, e^{-(\omega-\omega_i)^2/w_i^2} }
\end{equation}
with the Fourier-representation
\begin{equation}
\label{eg:garesF}
{\cal R}(Q,\omega) = \sum_{i=1}^{N} {\int \frac{a_i \, e^{-(w_i \, t)^2/4}}{ 2 \pi^{3/2} \, w_i }\, \cos([\omega-\omega_i]\,t) d\omega}
\end{equation}
typically the components of the Gaussian representations have a rather limited extension in
the time domain such that the numerical Fouriertransform of the product with the intermediate
scattering function can be restricted to a corresponding maximum time, thus yielding an 
efficient computation of the expected experimentally observed spectrum
\begin{equation}
\label{eq:conv1}
I(Q,\omega) = \sum_{i=1}^{N} { \int_0^{t_{\max} = W_t/w_i} {
\frac{a_i \, e^{-(w_i \, t)^2/4}}{ 2 \pi^{3/2} \, w_i } \, \cos([\omega-\omega_i]\, t) \, S(Q,t) \, dt 
}}
\end{equation}
where the coefficient $W_t$ used for the upper time limit is selected such that the integral
contributions beyond that point are sufficiently reduced, 
%here a value of 5 is used sich that the integrand is reduced with a factor of less than $2 \times 10^{-3}$ 
here a value of 9 is used sich that the integrand is reduced with a factor of less than $2 \times 10^{-9}$ 
due to the resolution
factor (if reqired the factor may easily be increased without causing too much extra computational effort). 
The procedure as outline above captures the salient features use for fitting the results,
however, to cope with effects pertaining to the finite width of one energy/frequency bin
$\Delta \omega$, which will play a role if broadening effects in the range of $\Delta \omega$
are to be considered, the final form of the method takes account for these effects by the
following modifications.
1. the resolution is fitted with the integral of the Gaussian over the channel width 
%
$ \sum_{i=1}^{N} { a_i \{ \sqrt{\pi/4}\, w_i/\Delta \omega \} \,(
{\rm erf}([2\omega+\Delta \omega]/2w_i)-{\rm erf}([2\omega-\Delta\omega]/2w_i)) $.
2. when computing the estimation of the measured intensity $I(Q,\omega)$ the same
integration over a channel width is performed, which effectively is achieved by
replacing Eq. \ref{eq:conv1} by
\begin{equation}
\label{eq:conv1}
I(Q,\omega) = \sum_{i=1}^{N} { \int_0^{t_{\max} = W_t/w_i} {
\frac{a_i \, e^{-(w_i \, t)^2/4}}{ 2 \pi^{3/2} \, w_i } \, 
\frac{ 
\sin(\{-[\omega-\omega_i]+\Delta\omega/2\}\, t)+\sin(\}[\omega-\omega_i]+\Delta\omega/2\}\, t)
}{
t \, \Delta\omega
} 
\, S(Q,t) \, dt 
}}
\end{equation}
 
%        strex_kernel_d0= 
%     *  sumtf * 
%     *  exp(-1d0/4d0*(str_delta*t)**2) * 
%     *  (sin(-t*Omega+0.5d0*t*xwidth)+sin(t*Omega+0.5d0*t*xwidth))/    !! this replaces cos(t*Omega)
%     *  (t*xwidth) * str_delta                                         !! in order to yield the integral
%                                                                       !! over one channel width in omega
This ensures proper treatment of spectral features with variations within the 
bin width of the histogramm of count results. For virtually any conceivable model for the
intermediate scattering function $S(Q,t)$ the numerical integration using an adaptive
method is reliable and fast. Due to the inclusion of the resolution convolution any
cutoff effects are avoided as well as the typical artefacts of replacing the integration 
by an FFT sum. 

Figure \ref{fig:resex} illustrates the representation of the IN5 resolution by two
Gaussians and a small constant background. The background is removed by using the
two Gaussians only fort convolution.
%fffffffffffffffffffffffffffffffffffffffffffffffffffffffffffffffffffffffffff
\begin{figure}[ht]
	\centering
  \includegraphics[width=0.5\textwidth, angle=90]{resolution_example.eps}
	\caption{IN5 resolution for a wavelength of $15\,{\rm \AA}$ at a scattering angle of $110^0$.
                 The data are the scattering from a 1~mm vanadium plate. The lines correspond to 
                 a model representation with two Gaussian (channelwise integrated) plus constant
                 backgound, which is not considered as part of the resolution function during the 
                 further data treatment.
                 The right intensity scale corresponds to the logarithmic plot (upper curve).}
	\label{fig:resex}
\end{figure}
%ffffffffffffffffffffffffffffffffffffffffffffffffffffffffffffffffffffffffffff  

Whereas symmetric resolution functions as those from IN5 (ILL) or TOFTOF (FRMII) may be satisfactory
modelled by 2 or 3 Gaussians, asymmetric lines as those from BASIS (SNS) need more. 
%
%


\section{References}


\bibliographystyle{plain.bst}
          a model representation with two Gaussian (channelwise integrated) plus constant
                 backgound, which is not considered as part of the resolution function during the 
                 further data treatment.
                 The right intensity scale corresponds to the logarithmic plot (upper curve).}
	\label{fig:resex}
\end{figure}
%ffffffffffffffffffffffffffffffffffffffffffffffffffffffffffffffffffffffffffff  

Whereas symmetric resolution functions as those from IN5 (ILL) or TOFTOF (FRMII) may be satisfactory
modelled by 2 or 3 Gaussians, asymmetric lines as those from BASIS (SNS) need more. 
%
%


\section{References}


\bibliographystyle{plain.bst}

\bibliography{monk12,biehl12,jsmith12,volino12,sears12,biodyn1,HYDROPRO_Harvey}




\end{document}
