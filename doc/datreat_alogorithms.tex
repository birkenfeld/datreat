\documentclass[12pt]{article}
%%%%%%%%%%%%%%%%%%%%%%%%%%%%%%%%%%%%%%%%%%%%%%%%%%%%%%%%%%%%%%%%%%%%%%%%%%%%%%%%%%%%%%%%%%%%
\def\TheAuthor{Michael Monkenbusch}                           % Author name
\def\TheTitle{datreat algorithms}                             % Title of contribution
\def\TheHeading{datreat algorithms}                           % Short running title
\def\TheInstitute{FZJ JCNS-1}                                 % Institute
\def\TheAddress{FZJ}                                          % Place e.g. University ...
\def\ThePart{}                                                % Part of book
\def\TheChapter{I}                                            % Chapter of contribution
%%%%%%%%%%%%%%%%%%%%%%%%%%%%%%%%%%%%%%%%%%%%%%%%%%%%%%%%%%%%%%%%%%%%%%%%%%%%%%%%%%%%%%%%%%%%

\usepackage{mylabbook}                                        % Style file for layout, derived from Ferienschule_07.sty
\usepackage{float}                                            % Figure placement float
\usepackage{etex}
\usepackage{epsfig,color}
\usepackage{makeidx}
\usepackage{cases}

% General translations
\newcommand{\Aa}{\rm{\AA}}
%
%%%%%%%%%%%%%%%%%%%%%%%%%%%%%%%%%%%%%%%%%%%%%%%%%%%%%%%%%%%%%%%%%%%%%%%%%%%%%%%%%%
% Auxiliaries for theory description
%%%%%%%%%%%%%%%%%%%%%%%%%%%%%%%%%%%%%%%%%%%%%%%%%%%%%%%%%%%%%%%%%%%%%%%%%%%%%%%%%%
% Headersection
\newcommand{\thname}[2]{{\Large \bf #1} \vskip 15pt {#2} \vskip 20pt}

% Parameter Section
\newcommand{\parameters}{\vskip 15pt {\bf Parameters:} \vskip 5pt \begin{tabular}{lllll} \hline name & symbol & meaning & units & fit \\ \hline}
\newcommand{\eparameters}{\hline \end{tabular}}
\newcommand{\parnam}[5]{#1 & #2 & #3 & #4 & #5\\}
% Description section

\newcommand{\thdescript}[1]{{\vskip 15pt \bf General description} \vskip 10pt #1}

% Computation
\newcommand{\algorithm}[1]{\vskip 15pt {\bf Algorithm} \vskip 15pt \begin{eqnarray} #1 \end{eqnarray}}

% Expected Fileparameters
\newcommand{\expectedfileparameters}{\vskip 15pt {\bf Expected file parameters:} \vskip 5pt \begin{tabular}{lllll} \hline name & symbol & meaning & units & optional \\ \hline}
\newcommand{\eexpectedfileparameters}{\hline \end{tabular}}
\newcommand{\efileparnam}[5]{#1 & #2 & #3 & #4 & #5\\}

% created File Parameters
\newcommand{\fileparameters}{\vskip 15pt {\bf Associated file parameters:} \vskip 5pt \begin{tabular}{llll} \hline name & symbol & meaning & units \\ \hline}
\newcommand{\efileparameters}{\hline \end{tabular}}
\newcommand{\fileparnam}[4]{#1 & #2 & #3 & #4 \\}


%%%%%%%%%%%%%%%%%%%%%%%%%%%%%%%%%%%%%%%%%%%%%%%%%%%%%%%%%%%%%%%%%%%%%%%%%%%%%%%%%%%%%%%%%%%
%%%%%%%%%%%%%%%%%%%%%%%%%%%%%%%%%%%%%%%%%%%%%%%%%%%%%%%%%%%%%%%%%%%%%%%%%%%%%%%%%%%%%%%%%%%
%%%%%%%%%%%%%%%%%%%%%%%%%%%%%%%%%%%%%%%%%%%%%%%%%%%%%%%%%%%%%%%%%%%%%%%%%%%%%%%%%%%%%%%%%%%
% \makeindex
\begin{document}
% \MakeTitel           %%% Displays title, author name, etc.
% \tableofcontents     %%% Displays table of contents
% \newpage
\setlength{\parskip}{0.3cm}
%%%%%%%%%%%%%%%%%%%%%%%%%%%%%%%%%%%%%%%%%%%%%%%%%%%%%%%%%%%%%%%%%%%%%%%%%%%%%%%%%%%%%%%%%%%

\section{average}
\label{sec:average}


\begin{enumerate}
\item Collect all data points fron different records into one linear array (x,y,error).
\item Sort that array to increasing error.
\item \label{av:it} Start with the (still left \emph{not used}) lowest error entry 
and look for all other points within 
a x-distance given by \emph{xcatch} (relative to ${x}$ or absolute) (index \emph{j}).
\item collect the information to create a new average point (index \emph{i}) using error-weighting:
${w_i=[ \sum_j {1/y_{err}^2}]}$ \\
${x_i=[(1/w_i) \sum_j {x_i/y_{err}^2}]}$ \\
${y_i=(1/w_i)[\sum_j {y_i/y_{err}^2}]}$ \\
${y_{err,i} = \sqrt{1/w_i}}$.
\item flag the selected initial and the selected \emph{neighbouring} points as \emph{used}.
\item Iterate by returning to point \ref{av:it} until all points are used.
\item Finally sort result array according to x.

\end{enumerate}

The collection range is controled by the value of \emph{xcatch} and the options
\emph{absolute} or \emph{relative} (=default).

For NSE type or other relaxation function type data normally \emph{relative} is the
approriate choice, \emph{xcatch} may tyically be between 0.05 and 0.5.


\end{document}
