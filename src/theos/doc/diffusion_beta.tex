\documentclass[12pt]{article}
\usepackage[titletoc]{appendix}
\usepackage{mathtools}
\usepackage{amsmath} 
\usepackage{hyperref}
\usepackage{graphicx}
\usepackage{epstopdf}
\usepackage{relsize}
\usepackage{fancyvrb}

\usepackage[firstpage]{draftwatermark}
\SetWatermarkScale{2}
\SetWatermarkLightness{0.7}

% for Verbatim
\fvset{formatcom=\color{blue}}
\fvset{fontsize=\relsize{-1}}

% %%%%%%%%%%%%%%%%%%%%%%%%%%%%%%%%%%%%%%%%%%%%%%%%%%%%%%%%%%%%%%%%%%%%%%%%%%%%%%

\title{NSE background subtraction issues}
\author{M. Monkenbusch 
JCNS, Forschungszentrum J\"uelich, Germany}
\date{\today}

% %%%%%%%%%%%%%%%%%%%%%%%%%%%%%%%%%%%%%%%%%%%%%%%%%%%%%%%%%%%%%%%%%%%%%%%%%%%%%%
\newcommand{\myspace}{\vskip 0.4cm}

% %%%%%%%%%%%%%%%%%%%%%%%%%%%%%%%%%%%%%%%%%%%%%%%%%%%%%%%%%%%%%%%%%%%%%%%%%%%%%%
\begin{document}
\title{diffusion\_beta}
%----------------------
%
Problem solved:
make a consistent connection of long time normal diffusion as e.g. measured by NMR to
some sublinear diffusion regime.
\myspace
Conditions
\begin{enumerate}
\item Gaussian approximation is valid
\item sub/super-linear diffusion at short time is described by $\langle r^2(t) \rangle \propto t^\beta$
\item at some transition time $t^*$ (tstar) the short time diffusion smoothly meets the long time diffusion
\item the condition that value and slope of  $\langle r^2(t) \rangle$ match at $t*$ implies an offset for
the mean square displacement in the long time behaviour.
\item besides the streching exponent $\beta$, tstar ($t^*$) is the only additional free parameter.
\end{enumerate}

After a few operation we the arrive at:
\begin{equation}
\label{eq:difbeta1}
 \langle r^2(t<t^*) \rangle = 6\,{\frac {{t}^{\beta}{{\it t^*}}^{-\beta+1}{\it D_{nmr}}}{\beta}} =
                              6\,{\frac {(t/t^*)^{\beta}\;({{\it t^*}}{\it D_{nmr})}}{\beta}}
\end{equation}
and
\begin{equation}
\label{eq:difbeta2}
 \langle r^2(t>t^*) \rangle = 6\,{\frac {{\it D_{nmr}}\, \left( \beta\,t-\beta\,{\it t^*}+{\it t^*}
 \right) }{\beta}}
= 
6\,D_{nmr}[t + t^* (1/\beta -1)] 
\end{equation}
the intermediate scattering (factor) then follows immediately from the Gaussian approximation.
\begin{equation}
S(Q,t) = \exp(-q^2 \,  \langle r^2(t) \rangle /6)
\end{equation}


\end{document}
