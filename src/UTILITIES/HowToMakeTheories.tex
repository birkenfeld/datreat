\documentclass{article}
\usepackage{graphicx}
\usepackage{float}
\usepackage{amsmath}
\usepackage[section]{placeins}



%% !!!!!!!!!! Please do not change the general layout !!!!!!!!!!!

%%%%%%%%%%%%% Statement of author
%%%%%%%%%%%%% Add appropriate text
%%%%%%%%%%%%% Template written by R. G. Winkler, Theorie II
%%%%%%%%%%%%% Modified by K. Tillmann, IFF-IMF

\def\TheAuthor{Michael Monkenbusch}                           % Author name
\def\TheTitle{Labbook Notes}                                  % Title of contribution
\def\TheHeading{Labbook Notes}                                % Short running title
\def\TheInstitute{JCNS}                                       % Institute
\def\TheAddress{FZJ}                                          % Place e.g. University ...
\def\ThePart{}                                                % Part of book
\def\TheChapter{I}                                            % Chapter of contribution

%% NOTE !!!!!!!!
%% If a statement exceeds one line and a line break at a
%% particular position is desired use \newline instead of \\
%%

\usepackage{mylabbook}                                        % Style file for layout, derived from Ferienschule_07.sty
\usepackage{float}                                            % Figure placement float
%\usepackage{fancybox}


\usepackage{etex}
%\usepackage[pdftex]{color}

%% \usepackage[pdftex,dvips]{graphicx}

\usepackage{epsfig,color}
\usepackage{makeidx}

\usepackage{cases}


\begin{document}
\section{Define the problem}
\label{sec:problem}

The model to be used for data-fitting has to be selected and expressed in terms
of some formula or procedures.

\begin{itemize}
\item Identify parameters that are to be fitted, they are common to all data records that enter
      the analysis. Select names.
\item Identify parameters that are associated to the data records and possibly discriminate them,
      e.g. q-value, temperature, $\cdots$.
\item What is the ordinate (x-value)?
\end{itemize}


\section{Programming Language}
\label{sec:proglang}

The programming language for \emph{datreat} theories is \emph{fortran}.
It is recommended (but not mandatory) to use modern (2003) language features.

Most of the work to setup the theorie function in \emph{fortran} is done by using the
tmeplate generator (see below). 
However, the contents in terms of a proper fromula expression or more sophisticated 
procedure muts be supplied as input to the template generator or edited at a later stage
into the theorie function source code. 

\section{Use the template generator}
\label{sec:template}

The template generator converts a template input containing the information on 
parameters and names etc. into a theory function source code.
Note that names are restricted to a length of 8 characters.
The template input is explained using the following example:

\small
\begin{verbatim}
001  #THEORY locrep
002          generalized local reptation expression
003          along the lines of deGennes, but with finite summation of integrals
004          and lenthscale, timescale and fluctuation ratio as parameters
005  #CITE   P.G. De Gennes. Coherent scattering by one reptating chain. 
006          Journal de Physique, 1981, 42 (5), pp.735-740. <10.1051/jphys:01981004205073500>     
007  #PARAMETERS
008          ampli            ! prefactor 
009          b                ! fluctuation intensity (relative)
010          a                ! length scale
011          tau              ! timescale
012          lz               ! total length
013  #RECIN-PARAMETERS    (allow 8 chars for name, start only beyond 8 with default)
014          q            0   ! q-value    default value
015  #RECOUT-PARAMETERS
016  #VARIABLES
017       double precision   :: t
018  #IMPLEMENTATION
019       t  = x              ! since we prefer to call the independent variable t, x must be copied to t
020       th = ampli * local_reptation(q*a, t/tau, lz)
021  
022  #SUBROUTINES
023    function local_reptation(q, t, L) result(val)
024      implicit none
025      double precision, intent(in)   :: q, t, L
026      double precision               :: val
027      double precision, parameter    :: sqp = sqrt(4*atan(1d0))
028  
029      val = 0.72D2 * (sqrt(t) * q ** 4 * exp((-0.2D1 * L * q ** 2 * t - 
030       #0.3D1 * L ** 2) / t / 0.12D2) / 0.36D2 + sqrt(0.3141592653589793D1
031       #) * (q ** 2 * t / 0.3D1 + L) * q ** 4 * exp(t * q ** 4 / 0.36D2) *
032       # (-erfc((q ** 2 * t + 0.3D1 * L) * t ** (-0.1D1 / 0.2D1) / 0.6D1) 
033       #+ erfc(sqrt(t) * q ** 2 / 0.6D1)) / 0.72D2 - sqrt(t) * q ** 4 / 0.
034       #36D2) * B / q ** 4 * 0.3141592653589793D1 ** (-0.1D1 / 0.2D1) / L 
035       #+ 0.72D2 * (A * exp(-q ** 2 * L / 0.6D1) * sqrt(0.3141592653589793
036       #D1) + (A * L * q ** 2 / 0.6D1 - A) * sqrt(0.3141592653589793D1)) *
037       # 0.3141592653589793D1 ** (-0.1D1 / 0.2D1) / q ** 4 / L
038  
039  
040    end function local_reptation
041  #END

\end{verbatim}
\normalsize

The \#KEYWORD lines start the different input groups.

\begin{itemize}
\item line 001:  \#THEORY locrep. Give a name (here: \emph{locrep}) to the theory.
\item lines 002-004: description of the theory
\item lines 005-006: citation 
\item lines 007-012: the fit parameters, list of names  and ! explanations
\item lines 013-014: parameters associated with the data records, here q. Name default value ! expalantion.
\item line 015: output parameters that are added to the record parameter section (here empty)
\item lines 016-017: fortran variable definition for any additional variables to be used in the theory
       (not needed for parameters, they are automatically created).
\item lines 018-020: implementation: fortran code to compute the theory
\item lines 022-040: any function and subroutines that are needed to do the computation

\end{itemize}

The theorie is then created by 

\emph{template\_generator ${<}$ template}

which creates a file \emph{out.f90} with the source code of the function.
Depending on the completenes of the template used this file may or may not need some further
finale editing to fill the gaps.
After that has been completed, copy it to a unique name in ./src/theos/ , i.e. \emph{th\_locrep.f90}
in this example.


\section{Finish}
\label{sec:finish}

Goto the \emph{./datreat/src} and type \emph{make}. This should incorporate the new
theory. If the make is not sucessful you may need some debugging or syntax fixes.
In that case edit the \emph{th\_locrep.f90} and retry make. 


\section{Test and Use}
\label{sec:testuse}

After sucessful \emph{make} the new theory should be listed if datreat is started
and the commend \emph{th} is entered. \emph{th thname} then will yield the information
on parameters etc. (here thname=\emph{locrep}).

To use it give the command \emph{ac thname} and then use \emph{cth} to enter parameter values.

\section{Code Generated}
\label{sec:codegen}


\tiny
\begin{verbatim}

 FUNCTION th_locrep(x, pa, thnam, parnam, npar,ini, nopar ,params,napar,mbuf)
!================================================================================
!  generalized local reptation expression along the lines of deGennes, but with finite summation of integrals and lenthscale, timescale and fluctuation ratio as parameters
!  Journal de Physique, 1981, 42 (5), pp.735-740. <10.1051/jphys:01981004205073500>
      use theory_description 
      implicit none 
      real    :: th_locrep
      character(len=8) :: thnam, parnam (*) 
      real    :: pa (*) 
      real    :: x , xh
      integer :: mbuf, nparx, ier, ini, npar, iadda
      integer, intent(inout) :: nopar       
      character(len=80), intent(inout) :: napar(mbuf) 
      real, intent(inout) :: params(mbuf) 
     
      double precision, parameter :: Pi = 4*atan(1d0)
      integer                     :: actual_record_address
     
! the internal parameter representation 
     double precision :: ampli      ! prefactor                                                                       
     double precision :: b          ! fluctuation intensity (relative)                                                
     double precision :: a          ! length scale                                                                    
     double precision :: tau        ! timescale                                                                       
     double precision :: lz         ! total length                                                                    
! the recin parameter representation 
     double precision :: q          ! q-value    default value                                                        
! the reout parameter representation 
 
     double precision :: th
 
     double precision   :: t
!
! ----- initialisation ----- 
    IF (ini.eq.0) then     
       thnam = 'locrep'
       nparx =        5
       IF (npar.lt.nparx) then
           WRITE (6,*)' theory: ',thnam,' no of parametrs=',nparx,' exceeds current max. = ',npar
          th_locrep = 0
          RETURN
       ENDIF
       npar = nparx
! >>>>> describe theory with >>>>>>> 
       idesc = next_th_desc()
       th_identifier(idesc)   = thnam
       th_explanation(idesc)  = " generalized local reptation expression along the lines of deGennes, but with finite summation of integrals and lenthscale, timescale and fluctuation ratio as parameters"
       th_citation(idesc)     = " Journal de Physique, 1981, 42 (5), pp.735-740. <10.1051/jphys:01981004205073500>"
!       --------------> set the parameter names --->
        parnam ( 1) = 'ampli   '  ! prefactor                                                                       
        parnam ( 2) = 'b       '  ! fluctuation intensity (relative)                                                
        parnam ( 3) = 'a       '  ! length scale                                                                    
        parnam ( 4) = 'tau     '  ! timescale                                                                       
        parnam ( 5) = 'lz      '  ! total length                                                                    
! >>>>> describe parameters >>>>>>> 
        th_param_desc( 1,idesc) = "prefactor" !//cr//parspace//&
        th_param_desc( 2,idesc) = "fluctuation intensity (relative)" !//cr//parspace//&
        th_param_desc( 3,idesc) = "length scale" !//cr//parspace//&
        th_param_desc( 4,idesc) = "timescale" !//cr//parspace//&
        th_param_desc( 5,idesc) = "total length" !//cr//parspace//&
! >>>>> describe record parameters used >>>>>>>
        th_file_param(:,idesc) = " " 
        th_file_param(  1,idesc) = "q        > q-value    default value"
! >>>>> describe record parameters creaqted by this theory >>>>>>> 
        th_out_param(:,idesc)  = " "
! 
        th_locrep = 0.0
 
        RETURN
     ENDIF
!
! ---- transfer parameters -----
      ampli    =      pa( 1)
      b        =      pa( 2)
      a        =      pa( 3)
      tau      =      pa( 4)
      lz       =      pa( 5)
! ---- extract parameters that are contained in the present record under consideration by fit or thc ---
      iadda = actual_record_address()
! >>> extract: q-value    default value
      xh =      0
      call parget('q       ',xh,iadda,ier)
      q        = xh
! 
! ------------------------------------------------------------------
! ----------------------- implementation ---------------------------
! ------------------------------------------------------------------
! 
     t  = x              ! since we prefer to call the independent variable t, x must be copied to t
     th = ampli * local_reptation(q*a, t/tau, lz)

     th_locrep = th
 
! ---- writing computed parameters to the record >>>  
 
 CONTAINS 
 
! subroutines and functions entered here are private to this theory and share its variables 
 
  function local_reptation(q, t, L) result(val)
    implicit none
    double precision, intent(in)   :: q, t, L
    double precision               :: val
    double precision, parameter    :: sqp = sqrt(4*atan(1d0))

    val = 0.72D2 * (sqrt(t) * q ** 4 * exp((-0.2D1 * L * q ** 2 * t -
     #0.3D1 * L ** 2) / t / 0.12D2) / 0.36D2 + sqrt(0.3141592653589793D1
     #) * (q ** 2 * t / 0.3D1 + L) * q ** 4 * exp(t * q ** 4 / 0.36D2) *
     # (-erfc((q ** 2 * t + 0.3D1 * L) * t ** (-0.1D1 / 0.2D1) / 0.6D1)
     #+ erfc(sqrt(t) * q ** 2 / 0.6D1)) / 0.72D2 - sqrt(t) * q ** 4 / 0.
     #36D2) * B / q ** 4 * 0.3141592653589793D1 ** (-0.1D1 / 0.2D1) / L
     #+ 0.72D2 * (A * exp(-q ** 2 * L / 0.6D1) * sqrt(0.3141592653589793
     #D1) + (A * L * q ** 2 / 0.6D1 - A) * sqrt(0.3141592653589793D1)) *
     # 0.3141592653589793D1 ** (-0.1D1 / 0.2D1) / q ** 4 / L


  end function local_reptation
 end function th_locrep
\end{verbatim}
\normalsize

\end{document}

